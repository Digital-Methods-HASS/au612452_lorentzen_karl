\documentclass{article}
\usepackage[utf8]{inputenc}

\title{Digital Methods: Learning Journal}
\author{Karl Lorentzen}
\date{Autumn 2019}

\begin{document}

\maketitle

\section{Today's Date}
\subsection{Thoughts / Intentions}
\subsection{Action}
\subsection{Results}
\subsection{Final Thoughts}

\pagebreak{}

\section{Introduction}
\textbf{Karl Lorentzen} This is a learning journal for the "Digital Methods for Historians" course, the second half of the theory-oriented method class at the history study at Aarhus University. The purpose of this journal is to chronicle my progression and actions as part of the aforementioned course, including the performance of various coding-related exercises and the progression of my work on the final project. The journal is divided by individual dates as sections, which will in turn be divided into subsections. The "Thoughts/Intentions" subsection will detail initial thoughts on what is to be done, the "Action" subsection details my work as I am doing it, the "Results" subsection details the outcome, while the "Final Thoughts" subsection details any final considerations on what has been accomplished and what will need to be done afterwards. The journal is written in source text, meaning that it also includes coding, providing additional exercise in coding. The source text is compiled into a pdf file which can then be published onto my own GitHub repository.

\section{31/10/2019}

\subsection{Action}

\begin{itemize}
\item Accidentally pressed "Enter" when using Regex, didn't know it automatically applied changes in the Regex box, important example of how a line break can also mess up the system.
\end{itemize}

\section{5/11/2019}
\subsection{Thoughts / Intentions}

\textbf {12:30pm}: Trying to do exercise for this week by converting stopword-list from Voyant to R, and vice versa. Can't remember the specifics, however.


\textbf{12:35pm}: Wrote another message to Adela asking for help on the conversion.


\textbf{12:45pm}: Took break from work on Voyant and R.

\section{7/11/2019}

\subsection{Thoughts/Intentions}

\textbf{8:15am}: Class today, want to find out how to do the exercise.

\subsection{Action}

\begin{itemize}
    \item Typed (",) in Regex, copy-pasted the substitute text up into the text string, then typed (") to remove the quotation marks at the beginning of the words. Don't know how to make line breaks.
    \item Moving on to convert from Voyant to R. Inserted the stopword-list from Excel-document, typed \b and inserted " in the substitution box, copy-pasted it into the text string box, then typed \$ in the regex box and inserted a , in the substitution box.
\end{itemize}

\subsection{Results}

\textbf{8:40am}: Did the exercise in converting stopword-list from Voyant to R and vice versa.

\subsection{Final Thoughts}

\textbf{11:55am}: Hope to get better at doing the next exercises.

\section{14/11/2019}

\subsection{Thoughts / Intentions}

\textbf{8:25am}: Class today on using Gitbash. Thinking about what to do with Danmarks Breve and which dataset to use.

\textbf{8:40am}: Adela suggests looking at different kinds of datasets to find out which one works best for us.

\subsection{Action}

\begin{itemize}
    \item Typed "ls" in Gitbash, shows various folders, hope they are all in order.
    \item Saved the data shell in my desktop, except it was in a cloud folder
    \item Moved the data shell to the regular Documents folder, easier to find in Gitbash now
    \item Created "thesis" folder in the Data shell folder using the mkdir command in
    \item Created text file inside the thesis folder using the "nano draft.txt" command in
    \item Deleted text file inside the thesis folder the "rm -i draft.txt" command
    \item Configured the username and email for Gitbash
\end{itemize}

\subsection{Results}

\textbf{09:53am} Learned rudimentary principles of Gitbash.

\subsection{Final Thoughts}

\textbf{11:03am} Learned a lot today, have a better sense of direction in the course and what to do.

\section{20/11/2019}
\subsection{Thoughts/Intentions}
\textbf{12:38pm}:  Doing exercises in RStudio.

\subsection{Action}

Tidyverse installation 
\begin{itemize}
\item Typed install.packages("tidyverse"), which caused RStudio to install large amounts of data.
\item \textbf{ERROR} Typing library(tidyverse) causes error message, says there is no tidyverse package.
\item Trying to install tidyverse through the packages tab rather than code
\item \textbf{ERROR} Same problem as before, though perhaps it's because I press escape when I try to bring out the > symbol.
\item Seems to be working after waiting long enough for the installation to finish.
\item Not sure how to proceed with the exercise itself, though.
\end{itemize}

\subsection{Final Thoughts}
\textbf{8:30pm} Hope to get help tomorrow with the exercise or at least understand the spirit of it.

\section{26/11/2019}
\subsection{Thoughts/Intentions}

\textbf{1:26pm} Trying to work with the conflict catalog spreadsheet for use in my project.

\section{28/11/2019}
\subsection{Action}
\begin{itemize}
    \item Started working on the script in RStudio with Adela, loaded the datasets into RStudio. Grouped the total population stats by year and calculated the mean. Then assigned the years to 100-year-long intervals, as in centuries.
\end{itemize}

\section{6/12/2019}
\subsection{Action}
\begin{itemize}
    \item Continued working on the script in RStudio with Adela, divided the wars by century.
\end{itemize}

\section{9/12/2019}
\subsection{Action}
\begin{itemize}
    \item With assistance from Adela, I combined the Conflict Catalogue and total world population datasets, created a new column as a function of the total fatalities divided by the total population in the respective century and then multiplied it by one hundred. That way, I created a new column that shows the total conflict-related fatalities as a percentage of the total world population in the century that the conflict took place in.
    \item Finished the script in RStudio with Adela.
    \item Tried to restructure the command lines in a more comprehensible order, accidentally broke up the correct flow in the coding, contacted Adela for help.
    \item Pinpointed the exact lines that gave error messages when activated.
\end{itemize}

\subsection{Results}

\textbf{4:00pm} Initially successful in finishing a functioning script, upon attempting to correct the command lines, I messed up the order in the script, forcing me to retrace my steps in order to pinpoint the error.

\subsection{Final Thoughts}

\textbf{5:26pm} Hoping to find a solution to my scripting errors relatively soon in order to move on with writing the paper for the project.

\section{10/12 2019}
\subsection{Action}
\begin{itemize}
    \item Upon removing superfluous coding, I was able to restore the script to functioning order.
\end{itemize}

\subsection{Final Thoughts}

\textbf{4:20pm} Will move on to start writing on the paper proper.

\section{18/12 2019}
\subsection{Thoughts/Intentions}

\textbf{1:27pm} Today, I intend to continue writing on the paper, still need to write sections on the credibility of my sources for my datasets.

\subsection{Action}
\begin{itemize}
    \item Added more information on the Conflict Catalog itself and the credentials of Peter Brecke as a researcher with the relevant experience, as well as the total world population dataset from Our World in Data.
    \item Added citations for the relevant websites and then I also corrected the new and previously existing citations according to the guidelines established in the special intro-week regarding digital archives for historians, including adding the various subsections of the websites. These then need to be followed up by the actual URLs.
\end{itemize}

\subsection{Results}
\textbf{2:27pm} I have now corrected the citations so that they include both the title of the page itself, the subsection(s), the name of the website itself, the date it was last checked, and then finally the URL.

\subsection{Final Thoughts}
\textbf{2:38pm} Will also need to upload this updated version of the learning journal to my repository in the Digital Methods group on GitHub.

\end{document}
